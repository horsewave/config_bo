% ---------------
% Description:write your Description
% Editor:Bo Ma
% date:2019/10/21
% Email:mabo@ihep.ac.cn
% Tel:010-88235869
% Cell:15210606357
%
% Version:v1.0
% ---------------
%
% \documentclass[a4paper]{article}

%if have chinese,use ctexart
\documentclass[a4paper]{ctexart}
% make the section name on the left, the default is center
\ctexset{section/format=\Large\bfseries}

\usepackage[utf8]{inputenc}
\usepackage[T1]{fontenc}
\usepackage{textcomp}
\usepackage{amsmath, amssymb}
\usepackage{xcolor}

% % figure support
\usepackage{import}
\usepackage{xifthen}
\usepackage{pdfpages}
\usepackage{transparent}

\title{My title}
\author{Bo Ma}

\begin{document}

\maketitle
%\begin{abstract}
%\end{abstract}
\section{一般步骤}%
\label{sec:gene-ste}
\begin{enumerate}
  \item 生成配置文件\colorbox{yellow}{ Doxyfile: doxygen (-s) -g}
  \item \begin{verbatim}
      建立目录结构, 比如Doxyfile文件 \doc文件夹(输出)\src文件夹(放代码) ->三者在同一目录下
    \end{verbatim} 
  \item 根据情况设置配置文件中的参数(关键)
  \item 运行 {\colorbox {yellow} {doxygen Doxyfile}} 生成大概结构
  \item 针对每个代码文件写doxygen注释
  \item 继续运行 {\colorbox {yellow} {doxygen Doxyfile}}
\end{enumerate}


\section{配置文件Doxyfile设置}%
\begin{verbatim}

DOXYFILE_ENCODING      = UTF-8
PROJECT_NAME           = "Project Name"
PROJECT_NUMBER         = 1.0
PROJECT_BRIEF          = "This is a brief descript."
EXTRACT_ALL            = YES
OUTPUT_DIRECTORY       = "./doc"
OUTPUT_LANGUAGE        = Chinese
FULL_PATH_NAMES        = NO
WARN_LOGFILE           ="./doc/build.log"
INPUT                  ="./src"
FILE_PATTERNS          =
SOURCE_BROWSER         = NO
GENERATE_LATEX         = NO
CALL_GRAPH             = YES
CALLER_GRAPH           = YES
UML_LOOK               = YES
RECURSIVE              = YES
 
\end{verbatim}


\subsubsection{文件注释}%
\label{ssub:wen_jian_zhu_shi_}

\begin{verbatim}
/**
    * Copyright (c) 2017, All rights reserverd.
    * 
    * @file $file$
    * @brief $brief$
    * Details.
    * 
    * @author  $author$,$email$
    * @date  $yy$-$mm$-$dd$
    * @version   $maj$.$min$
    *
    *************************************************/

\subsubsection{命名空间}%
\label{ssub:ming_ming_kong_jian_}


\begin{verbatim}
    /**
    * @brief $brief$
    * Details.
    **/

\end{verbatim}
   

\subsubsection{结构体/枚举}%
\label{ssub:jie_gou_ti_mei_ju_}


\begin{verbatim}
    /** 
    * @brief $brief$
    * Details.
    **/

\end{verbatim}
   
\subsubsection{类注释}%
\label{ssub:lei_zhu_shi_}
\begin{verbatim}

    /**
    * @brief $brief$
    * Details.
    **/

\end{verbatim}
   
\subsubsection{函数注释}%
\label{ssub:han_shu_zhu_shi_}


   \begin{verbatim}
    /** 
    * @brief $brief$-$test$
    * Details.
    * @param $param1$ : $param1_detail$
    * @param $param2$ : $param2_detail$
    * @return $return_detail$
    *        -<em>false</em> fail
    *        -<em>true</em>  succeed
    * @retval $return_note$
    * @deprecated $deprecated$
    * @see $see$
    * @pre $pre$
    **/

   \end{verbatim}
      
\section{关键字收集}%
\label{sec:guan_jian_zi_shou_ji_}

\begin{itemize}
  \item {\color{green}@author}  作者
  \item {\color{green}@brief}  摘要
  \item {\color{green}@version}  版本号
  \item {\color{green}@date}  日期
  \item {\color{green}@file}文件名,可以默认为空,DoxyGen会自己加
  \item {\color{green}@class}  类名
  \item {\color{green}@param         }函数参数
  \item {\color{green}@return}  函数返回值描述
  \item {\color{green}@exception}  函数抛异常描述
  \item {\color{green}@warning}  函数使用中需要注意的地方
  \item {\color{green}@remarks}备注
  \item {\color{green}@see}  see also字段
  \item {\color{green}@note}brief下空一行后的内容表示详细描述,但也可以不空行用note表示
  \item {\color{green}@par}  开始一个段落,段落名称描述由你自己指定,比如可以写一段示例代码
  \item {\color{green}@code}引用代码段
  \item {\color{green}@endcode}  引用代码段结束
  \item {\color{green}@pre}函数前置条件,比如对输入参数的要求
  \item {\color{green}@post}  函数后置条件,比如对系统状态的影响或返回参数的结果预期
  \item 
  \
  \item {\color{green}@}am[in|out]     参数名及其解释
  \item {\color{green}@exception}  用来说明异常类及抛出条件
  \item {\color{green}@return}对函数返回值做解释
  \item {\color{green}@note}表示注解,暴露给源码阅读者的文档
  \item {\color{green}@remark}  表示评论,暴露给客户程序员的文档
  \item {\color{green}@since}  表示从那个版本起开始有了这个函数
  \item {\color{green}@deprecated}  引起不推荐使用的警告
  \item {\color{green}@see}  表示交叉参考
 
\end{itemize}
@author       作者


\end{document}


